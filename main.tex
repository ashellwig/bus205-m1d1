\documentclass[12pt]{article}

  \usepackage[english]{babel}
  \usepackage{hyperref}
  \usepackage{fancyhdr}
  \usepackage[dvipsnames]{xcolor}
  \usepackage{listings}
  \usepackage{parcolumns}
  \usepackage{algorithm}
  \usepackage{algorithmicx}
  \usepackage{algpseudocode}
  \usepackage{enumitem}
  \usepackage{geometry}
  \usepackage{soul}
  \usepackage{graphicx}
  \usepackage{enumitem}
  \usepackage{csquotes}
  \usepackage{bookmark}
  \usepackage{mdframed}
  \usepackage{mathtools}
  \usepackage{amsmath}
  \usepackage{amsthm}
  \usepackage[toc]{appendix}
  \usepackage[
    backend=biber,%
    style=apa%
  ]{biblatex}

  % Bibliography Setup
  \addbibresource{main.bib}
  \newcommand{\CiteSection}[2]{%
    (\autocite{#1}, ~\S {#1})
  }

  % Theorem Environments
  \theoremstyle{definition}
  \newtheorem*{defn*}{Definition}
  \theoremstyle{plain}
  \newtheorem*{equ*}{Equation}

  % Definitions for Algorithmic Environments
  \algdef{SE}[VARIABLES]{GVariables}{EndGVariables}
    {\algorithmicvariables}
    {\algorithmicend\ \algorithmicvariables}
  \algnewcommand{\algorithmicvariables}{\textbf{global variables}}
  
  \algdef{SE}[VARIABLES]{LVariables}{EndLVariables}
    {\algorithmiclvariables}
    {\algorithmicend\ \algorithmiclvariables}
  \algnewcommand{\algorithmiclvariables}{\textit{local variables}}

  \renewcommand{\algorithmicrequire}{\textbf{Input:}}
  \renewcommand{\algorithmicensure}{\textbf{Output:}}
  \renewcommand\thealgorithm{}

  % Settings for math-mode
  \makeatletter
  \def\mathcolor#1#{\@mathcolor{#1}}
  \def\@mathcolor#1#2#3{%
    \protect\leavevmode
    \begingroup
      \color#1{#2}#3%
    \endgroup
  }
  \makeatother


  % Image Directory
  \graphicspath{ {screenshots/} }
  % Hyperlink Setup
  \hypersetup{
    colorlinks = true,
    urlcolor = blue,
    linkcolor = blue,
    citecolor = blue
  }
  % Page and Text Layout
  \pagestyle{fancy}
  \geometry{%
    a4paper,%
    top=15pt,%
    bottom=1in,%
    left=1in,%
    right=1in%
  }
  \setlength{\headheight}{15pt}

  \newenvironment{ldefinitions}
    {\left.\begin{aligned}}
    {\end{aligned}\right\rbrace}

  \title{%
    Module 1 Discussion 1%
    \large{Myths and Mythology}
  }
  \author{Ashton Hellwig}
  \date{\today}
  \rhead{HUM115 Module 1 Discussion 1}

\begin{document}
  \maketitle
  \tableofcontents
  \newpage


  \part{Initial Post}

    \section{Discussion Prompt}
      \begin{mdframed}
        \subsection{Overview}
          In this discussion we will examine what your belief of what a myth is
            and what purpose(s) it has.  You will need to read the Preface and
            Introduction to our textbook, (pgs. xi-xxi,) as a means to respond
            to this discussion.

        \subsection{Instructions}
          Compose a brief statement that clarifies your belief of what myth is
            and what purpose(s) it has. Give an example of some legend or tale
            from your experience and explain how the elements of that tale
            reinforced (or were intended to reinforce) particular customs or
            values. Please try to limit your contribution to an example that
            seems to have a purpose. Local ghost stories and/or odd strange
            events usually do not qualify. Compare your insights with at least
            two classmates.

          Be sure to proof-read your work (ALL rules of grammar, punctuation,
            spelling apply) and remember that I expect two well constructed
            paragraphs. I WILL deduct points for quick and shoddily written
            posts.
      \end{mdframed}

    \section{Response}
      \subsection{Myths: Purpose and Meaning}
        A myth to me is similar to folklore in that both categories tend to be
          orally passed-down to younger family members via their elders
          \autocite[p.~xiii]{textbook}. The difference between folklore and
          myths is in the plot and theme. Folklore tends to have some sort of
          main character which needs to overcome an obstacle and the story
          follows them. Myths tend to be more religious and/or spiritual in
          motivation -- such as ``\textit{The Flood Myth}''
          \autocite[p.~xiv]{textbook}. This is due to the fact that myths are
          most commonly used to explain \textit{``big-picture''}-type
          questions in human nature and history.

        I am currently finding it difficult to think of a myth which has had
          impact on my life \textit{personally}. This is most likely due to
          always, even as a child, I`ve focused on trying to have a more
          analytical view of most things (especially things which are hard
          to explain in natural terms). After reading some other experiences
          from classmates perhaps I will be reminded of something which happened
          while far younger that I am unable to now.


  \newpage
  \part{Responses}

    \section{Response 1}
      \begin{quote}
        Reply to \textbf{} (\textit{Post ID:})
      \end{quote}
      Placeholder

    \section{Response 2}
      \begin{quote}
        Reply to \textbf{} (\textit{Post ID: }) 
      \end{quote}
      Placeholder

  % Bibliography
  \newpage
  \nocite{textbook}
  \printbibliography[
    heading=bibintoc,
    title={Works Cited}
  ]
\end{document}
