\documentclass[12pt]{article}
  \usepackage[english]{babel}
  \usepackage{hyperref}
  \usepackage{fancyhdr}
  \usepackage[dvipsnames]{xcolor}
  \usepackage{listings}
  \usepackage{parcolumns}
  \usepackage{algorithm}
  \usepackage{algorithmicx}
  \usepackage{algpseudocode}
  \usepackage{enumitem}
  \usepackage{geometry}
  \usepackage{soul}
  \usepackage{graphicx}
  \usepackage{enumitem}
  \usepackage{csquotes}
  \usepackage{bookmark}
  \usepackage{mdframed}
  \usepackage{mathtools}
  \usepackage{amsmath}
  \usepackage{amsthm}
  \usepackage[toc]{appendix}
  \usepackage[
    backend=biber,%
    style=apa%
  ]{biblatex}

  % Bibliography Setup
  \addbibresource{main.bib}
  \newcommand{\CiteSection}[2]{%
    (\autocite{#1}, ~\S {#1})
  }


  % Image Directory
  \graphicspath{ {screenshots/} }
  % Hyperlink Setup
  \hypersetup{
    colorlinks = true,
    urlcolor = blue,
    linkcolor = blue,
    citecolor = blue
  }

  % Page and Text Layout
  \pagestyle{fancy}
  \geometry{%
    a4paper,%
    top=15pt,%
    bottom=1in,%
    left=1in,%
    right=1in%
  }
  \setlength{\headheight}{15pt}
  \title{%
    Unit 1 Discussion 1\\%
    \large{The Myth of the Amoral Business}
  }
  \author{Ashton Hellwig}
  \date{\today}
  \rhead{BUS205 Unit 1}

\begin{document}
  \maketitle
  
  \section{Initial Post}
    \begin{mdframed}
      \vspace{-10pt}
      \subsection*{Prompt}
      \begin{enumerate}
        \item \textbf{Myth of the Amoral Business}: What is the ‘myth of amoral business’? What evidence is
        there to support the existence of the myth; what evidence is there that refutes this perspective as
          myth? I encourage you to gather evidence of one or the other perspective from current news related press articles.
        \item \textbf{Law and Business Ethics}: Is the law a sufficient guide for determining the duties and
          obligations of business, a corporation or the individuals involved in the world of business? Why or
          why not?
      \end{enumerate}
    \end{mdframed}
  
    \subsection{Myth of the Amoral Business}
      The \textit{Myth of the Amoral Business} is -- according to DeGeorge, the stance that
        decrees business is only concerned about profit and should not be considered the
        type of activity that deals with right and wrong. DeGeorge believes that there is
        an inordinate amount of business which believe this -- stating that the actions
        of a firm hold no moral value. This is what is dubbed the ``Myth of the Amoral Business''.
        
      There is a great amount of evidence both supporting and refuting this ``Myth'' of the
        Amoral Business. First, we will discuss what \textit{does} support the fact that
        \textit{most if not all} businesses are more amoral than most individuals. As
        an example, look at the once-prime pharmaceutical company known as Valeant Pharmaceuticals.
        Rather than spend any money on research and development, they simply acted more like a
        bank specialized in mergers and acquisitions. Valeant would just purchase any company
        it could afford and use their R\&D to arbitrarily raise the prices of the drugs they
        were selling, as well as funneling dirty money through online pharmacies in a sort
        of money laundering scheme. Medicine is supposed to \textit{help} people. After
        what Valeant did, a simple pill to treat Wilson`s disease increased in price to the
        tune of nearly \$18,000 per month (from \$200/mo.). The CEO and chairman of Valeant at
        the time was not even the least bit concerned with the well being of his customers,
        only with the bottom line dictated to his shareholders.
        
      In terms of supporting the idea that morally relevant decisions are an unavoidable part
        of the business world -- as DeGeorge suggests -- the argument lies mainly on the fact that
        the Myth of the Amoral Business rests on faulty premises.
        
      \subsection{Law and Business Ethics}
        The law is really not the greatest guide in terms of determining the true moral
          responsibility of a business. For example, think about the situation in which
          Wells Fargo Bank, N.A. got into really hot water with their opening of extraneous
          banking accounts without permission of the account holders. This was done to
          improve their metrics in regards to how many accounts are opened per personal
          banker of a specific period of time. There are laws against opening accounts
          under an individual`s name without his or her permission \autocite{mclean_2017}.
    \newpage
  \section{Responses}

    \subsection{Response 1}
      \begin{quote}
        Reply to \textbf{} (\textit{Post ID:})
      \end{quote}
      Placeholder

    \subsection{Response 2}
      \begin{quote}
        Reply to \textbf{} (\textit{Post ID: }) 
      \end{quote}
      Placeholder

  % Bibliography
  \newpage
  \printbibliography[
    heading=bibintoc,
    title={Works Cited}
  ]
\end{document}
