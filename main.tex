\documentclass[12pt]{article}
  \usepackage[english]{babel}
  \usepackage{hyperref}
  \usepackage{fancyhdr}
  \usepackage[dvipsnames]{xcolor}
  \usepackage{listings}
  \usepackage{parcolumns}
  \usepackage{algorithm}
  \usepackage{algorithmicx}
  \usepackage{algpseudocode}
  \usepackage{enumitem}
  \usepackage{geometry}
  \usepackage{soul}
  \usepackage{graphicx}
  \usepackage{enumitem}
  \usepackage{csquotes}
  \usepackage{bookmark}
  \usepackage{mdframed}
  \usepackage{mathtools}
  \usepackage{amsmath}
  \usepackage{amsthm}
  \usepackage[toc]{appendix}
  \usepackage[
    backend=biber,%
    style=apa%
  ]{biblatex}

  % Bibliography Setup
  \addbibresource{main.bib}
  \newcommand{\CiteSection}[2]{%
    (\cite{#1}, ~\S {#2})\hspace{-4pt}
  }


  % Image Directory
  \graphicspath{ {screenshots/} }
  % Hyperlink Setup
  \hypersetup{
    colorlinks = true,
    urlcolor = blue,
    linkcolor = blue,
    citecolor = blue
  }

  % Page and Text Layout
  \pagestyle{fancy}
  \geometry{%
    a4paper,%
    top=15pt,%
    bottom=1in,%
    left=1in,%
    right=1in%
  }
  \setlength{\headheight}{15pt}
  \title{%
    Unit 1 Discussion 1\\%
    \large{The Myth of the Amoral Business}
  }
  \author{Ashton Hellwig}
  \date{\today}
  \rhead{BUS205 Unit 1}

\begin{document}
  \maketitle
  
  \section{Initial Post}
    \begin{mdframed}
      \vspace{-10pt}
      \subsection*{Prompt}
      \begin{enumerate}
        \item \textbf{Myth of the Amoral Business}: What is the ‘myth of amoral business’? What evidence is
          there to support the existence of the myth; what evidence is there that refutes this perspective as
          myth? I encourage you to gather evidence of one or the other perspective from current news related press articles.
          \begin{itemize}
              \item Textbook Pg. 3-5
          \end{itemize}
        \item \textbf{Law and Business Ethics}: Is the law a sufficient guide for determining the duties and
          obligations of business, a corporation or the individuals involved in the world of business? Why or
          why not?
          \begin{itemize}
              \item Textbook Pg. 5-9
          \end{itemize}
      \end{enumerate}
    \end{mdframed}
  
    \subsection{Myth of the Amoral Business}
      The \textit{Myth of the Amoral Business} is a phenomenon in which -- according to DeGeorge --
        businesses are simply only concerned about profit and should not concern themselves with the
        ``\textit{rightness}'' or ``\textit{wrongness}'' of its actions taken to achieve that
        profit \autocite{textbook}. DeGeorge goes on to state that those in business do not view
        themselves as unethical or immoral, but rather they hold the view that business is not
        to be concerned about ethics. While a firm may \textit{believe} that its actions are
        not even concerned with morality in nature, the fact is that every action taken by a
        firm -- just like every action taken by an individual -- holds moral value (positive or
        negative).

      As an example, we can look at the issue that took place two years ago around this time
        when Wells Fargo was caught instructing (and ignoring) the fact that employees were
        gaming (creating fake accounts to inflate personal sales metrics) as well as
        pinning (creating and assigning fake PIN numbers to customer`s accounts without
        their knowledge in order to impersonate them online and maintain said fake accounts)
        \autocite{mclean_2017}. In my opinion the \textbf{most} unethical thing about this
        situation was the way in which the Wells Fargo executives were handling it. They
        had fired over 1,000 low-level employees, paid no money out-of-pocket, and did not
        hold any repercussions on any higher-up employee while pushing $100\%$ of the blame
        on lower-level sales people. The individuals committing these actions \textit{knew}
        that what they were doing was immoral but because of the cut-throat competitiveness
        present in the company culture no one would take the reporting of the gaming
        seriously.

    \subsection{Law and Business Ethics}
      I do not believe that the law is a sufficient enough guide when it comes to a business`s
        determination of its obligations, ethically speaking. The law and morality do not
        always mirror one-another -- not all illegal actions are immoral and not all immoral
        actions are illegal \CiteSection{hellmers_2020}{III}. For example, laws which allowed
        the purchase and sale of slaves prior to the emancipation proclamation were $100\%$
        legal, but as \textit{we now know} this is an \textbf{incredibly} inhumane act. In
        some cultures honor killings are protected by law, and \textit{practically everyone}
        knows that murder, of any kind, is technically immoral.
        
      With the examples given above we can easily see how the law is an insufficient guide
        for businesses to base their moral actions on. If a certain morally-questionable
        action is allowed in the eyes of the law or a business takes advantage of a legal
        loophole in order to increase its profit, this is not \textit{amoral}, but rather
        \textit{immoral}.


%   \newpage
%   \section{Responses}

%     \subsection{Response 1}
%       \begin{quote}
%         Reply to \textbf{} (\textit{Post ID:})
%       \end{quote}
%       Placeholder

%     \subsection{Response 2}
%       \begin{quote}
%         Reply to \textbf{} (\textit{Post ID: }) 
%       \end{quote}
%       Placeholder


  % Bibliography
  \newpage
  \printbibliography[
    heading=bibintoc,
    title={Works Cited}
  ]
\end{document}
